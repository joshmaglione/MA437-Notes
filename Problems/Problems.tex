\documentclass[a4paper, 12pt]{article}

\usepackage{amsmath}
\usepackage{amsthm}
\usepackage{amssymb}
\usepackage{enumitem}
\usepackage{hyperref}
\hypersetup{
	colorlinks=true,
	linkcolor=blue,
	filecolor=blue,
	urlcolor=blue,
	citecolor=blue,
}
\usepackage[margin=3cm]{geometry}
\usepackage{mathpazo}
\usepackage{url}
\usepackage{subcaption}
\usepackage{tikz}
\usepackage{pgf}
\usepackage{longtable}
\usepackage{multirow}
\usepackage{graphicx}
\usepackage{cleveref}
\usepackage{bbm}
\usepackage{wrapfig}
\usepackage{mathrsfs}
\usepackage{afterpage}
\usepackage[svgnames]{xcolor}
\usepackage{marvosym}
\usepackage[framemethod=TikZ]{mdframed}
\usepackage{pgfplots}
\usetikzlibrary{3d}

\numberwithin{equation}{section}
\numberwithin{figure}{section}

\newtheorem{theorem}{Theorem}[section]
\newtheorem{thm}{Theorem}[section]
\newtheorem*{thm*}{Theorem}
\newtheorem*{con*}{Conjecture}
\newtheorem{lem}[thm]{Lemma}
\newtheorem{prop}[thm]{Proposition}
\newtheorem{cor}[thm]{Corollary}
\newtheorem{lemma}[thm]{Lemma}
\newtheorem{conj}[thm]{Conjecture}

\theoremstyle{definition}
\newtheorem{defn}[thm]{Definition}
\newtheorem{remark}[thm]{Remark}
\newtheorem{ex}[thm]{Example}
\newtheorem{quest}[thm]{Question}
\newtheorem{obs}[thm]{Observation}
\newtheorem{notation}[thm]{Notation}
\newtheorem{exercise}{Exercise}

\newenvironment{mybox}[1][]{%
\ifstrempty{#1}%
% if condition (without title)
{\mdfsetup{%
    frametitle={%
        %\tikz[baseline=(current bounding box.east),outer sep=0pt]
        %\node[anchor=east,rectangle,fill=RoyalBlue!80] {};
		}
    }%
% else condition (with title)
}{\mdfsetup{%
    frametitle={%
        \tikz[baseline=(current bounding box.east),outer sep=0pt]
        \node[anchor=east,rectangle,fill=RoyalBlue!80,text=white]
        {\strut #1};}%
    }%
}%
% Both conditions
\mdfsetup{%
    innertopmargin=10pt,linecolor=RoyalBlue!80,%
    linewidth=2pt,topline=true,%
    frametitleaboveskip=\dimexpr-\ht\strutbox\relax%
}
\begin{mdframed}[]\relax}{%
\end{mdframed}}

\newenvironment{myboxgreen}[1][]{%
\ifstrempty{#1}%
% if condition (without title)
{\mdfsetup{%
    frametitle={%
        %\tikz[baseline=(current bounding box.east),outer sep=0pt]
        %\node[anchor=east,rectangle,fill=RoyalBlue!80] {};
		}
    }%
% else condition (with title)
}{\mdfsetup{%
    frametitle={%
        \tikz[baseline=(current bounding box.east),outer sep=0pt]
        \node[anchor=east,rectangle,fill=ForestGreen!80,text=white]
        {\strut #1};}%
    }%
}%
% Both conditions
\mdfsetup{%
    innertopmargin=10pt,linecolor=ForestGreen!80,%
    linewidth=2pt,topline=true,%
    frametitleaboveskip=\dimexpr-\ht\strutbox\relax%
}
\begin{mdframed}[]\relax}{%
\end{mdframed}}

\newenvironment{myboxex}[1][1]{%
\vspace{0.25em}
\noindent \begin{minipage}{0.99\textwidth}
	\centering
	\begin{minipage}{#1\textwidth}
		\begin{myboxgreen}[]{}
}{%
		\end{myboxgreen}
	\end{minipage}
\end{minipage}
\vspace{0.25em}
}

\renewcommand{\leq}{\leqslant}
\renewcommand{\geq}{\geqslant}
\newcommand{\N}{\mathbb{N}}
\newcommand{\Z}{\mathbb{Z}}
\newcommand{\Q}{\mathbb{Q}}
\newcommand{\R}{\mathbb{R}}
\newcommand{\C}{\mathbb{C}}
\newcommand{\define}[1]{\textbf{\textit{#1}}}
\newcommand{\WEEK}[1]{%
\hfill Week #1

\vspace{-1em}

\begin{center}
	\rule{\textwidth}{2pt}
\end{center}
\vspace{0.5em}%
}

\setcounter{tocdepth}{2}

\allowdisplaybreaks

\title{Additional Problems}
\author{Joshua Maglione}
\date{\today}

\begin{document}

\maketitle

These problems are taken essentially directly from our main text: Bernard Kolman
and Robert E. Beck's textbook titled \emph{Elementary linear programming with
applications}.


\section*{Problem 1}

Show that the following hold.
\begin{enumerate}[label=$(\roman* )$]
    \item A hyperplane in $\mathbb{R}^n$ is a convex set.
    \item A close half-space in $\mathbb{R}^n$ is a convex set.
    \item An intersection of a finite collection of convex sets in
    $\mathbb{R}^n$ is convex.
    \item Suppose $f : \mathbb{R}^a \to \mathbb{R}^b$ is a linear
    transformation. If $S\subseteq \mathbb{R}^a$ is a convex set, show that the
    following set is convex:
    \[
        \left\{ f(v) ~\middle|~ v\in S \right\} \subseteq \mathbb{R}^b.
    \]
\end{enumerate}

\section*{Problem 2}

Consider the following LP

\begin{myboxex}[0.75]
    Maximize 
    \begin{align*}
        z &= 3x + 2y 
    \end{align*}
    subject to $x,y\geq 0$ and 
    \begin{align*}
        2x - y &\leq 6, \\
        2x + y &\leq 10.
    \end{align*}
\end{myboxex}
\begin{enumerate}[label=$(\roman* )$]
    \item Transform this problem to a problem n canonical form.
    \item For each extreme point of the new problem, identify the basic
    variables.
    \item Solve the given LP.
\end{enumerate}

\section*{Problem 3}

Consider the following simplex tableau that is missing its objective row and
column.

\begin{center}
    \begin{tabular}{|c|ccccccc|c|}
        \hline
        & $x_1$ & $x_2$ & $x_3$ & $x_4$ & $x_5$ & $x_6$ & $x_7$ & \\ \hline
        $x_4$ & $0$ & $0$ & $2$ & $1$ & $5/2$ & $0$ & $0$ & $6/7$ \\ 
        $x_1$ & $1$ & $0$ & $5$ & $0$ & $-3$ & $0$ & $-2$ & $2/7$ \\ 
        $x_6$ & $0$ & $0$ & $3$ & $0$ & $4$ & $1$ & $-4$ & $5/7$ \\ 
        $x_2$ & $0$ & $1$ & $0$ & $0$ & $3/2$ & $0$ & $0$ & $1/7$ \\ \hline
    \end{tabular}
\end{center}

Determine the departing variable if the entering variable is
\begin{enumerate}[label=$(\roman*)$]
    \item $x_5$;
    \item $x_3$;
    \item $x_7$.
\end{enumerate}

\section*{Problem 4}

Suppose $A\in \mathrm{Mat}_{m\times n}(\mathbb{R})$, $b\in \mathbb{R}^m$, and
$c\in \mathbb{R}^n$. Consider the following LP.

\begin{myboxex}[0.75]
    Maximize 
    \begin{align*}
        z &= c^{\top}x
    \end{align*}
    subject to $x\geq 0$ and 
    \begin{align*}
        Ax &\leq b. 
    \end{align*}
\end{myboxex}

Show that if $u$ and $v$ are feasible solutions, then the following point is a
feasible solution:
\[ 
    w = \dfrac{1}{3} u + \dfrac{2}{3} v.
\]

\section*{Problem 5}

Use one iteration of the Simplex Algorithm to obtain the next tableau from the following.
\begin{center}
    \begin{tabular}{|c|ccccc|c|}
        \hline 
        & $x_1$ & $x_2$ & $x_3$ & $x_4$ & $x_5$ & \\ \hline
        $x_3$ & $2/3$ & $0$ & $1$ & $3/5$ & $0$ & $3/2$ \\ 
        $x_2$ & $3/2$ & $1$ & $0$ & $1$ & $0$ & $5/2$ \\ 
        $x_5$ & $5$ & $0$ & $0$ & $2/9$ & $1$ & $2/3$ \\ \hline
        & $4$ & $0$ & $0$ & $-5$ & $0$ & $7/3$ \\ \hline
    \end{tabular}
\end{center}

\section*{Problem 6}

Consider the system of equations $Ax=b$ where
\begin{align*}
    A &= \begin{bmatrix}
        2 & 3 & 1 & 0 & 0 \\ 
        -1 & 1 & 0 & 2 & 1 \\
        0 & 6 & 1 & 0 & 3
    \end{bmatrix}, & b &= \begin{bmatrix}
        1 \\ 1 \\ 4
    \end{bmatrix}.
\end{align*}
Determine (with justification) whether each of the following points is a basic solution to the system:
\begin{align*}
    & \begin{bmatrix}
        1 \\ 0 \\ -1 \\ 1 \\ 0 
    \end{bmatrix}, & & \begin{bmatrix}
        0 \\ 2 \\ -5 \\ 0 \\ -1 
    \end{bmatrix}, & & \begin{bmatrix}
        0 \\ 0 \\ 1 \\ 0 \\ 1
    \end{bmatrix}.
\end{align*}

\section*{Problem 7}

Consider the following tableau that arose from solving an LP by the simplex method.
\begin{center}
    \begin{tabular}{|cccccc|c|}
        \hline 
        $x_1$ & $x_2$ & $x_3$ & $s_1$ & $s_2$ & $s_3$ & \\ \hline
        $1$ & $5$ & $2$ & $0$ & $0$ & $3$ & $20$ \\
        $0$ & $2$ & $4$ & $1$ & $0$ & $-4$ & $6$ \\
        $0$ & $2$ & $-1$ & $0$ & $1$ & $3$ & $12$ \\ \hline 
        $0$ & $-5$ & $-3$ & $0$ & $0$ & $3$ & $12$ \\ \hline
    \end{tabular}
\end{center}
\begin{enumerate}[label=$(\roman*)$]
    \item identify the basic feasible solution and basic variables in this
    tableau.
    \item Compute the next tableau using the simplex method.
    \item identify the basic feasible solution and basic variables in the new
    tableau from $(ii)$.
\end{enumerate}

\section*{Problem 8}

Use the Two-Phase Method to solve the following LP.

\begin{myboxex}[0.75]
    Maximize 
    \begin{align*}
        z &= x_1 + 2x_2 + x_3
    \end{align*}
    subject to $x\geq 0$ and 
    \begin{align*}
        3x_1 + x_2 - x_3 &= 15, \\
        8x_1 + 4x_2 - x_3 &= 50, \\
        2x_1 + 2x_2 + x_3 &= 20.
    \end{align*}
\end{myboxex}

\end{document}