\documentclass[a4paper, 12pt]{article}

\usepackage{amsmath}
\usepackage{amsthm}
\usepackage{amssymb}
\usepackage{enumitem}
\usepackage{hyperref}
\hypersetup{
	colorlinks=true,
	linkcolor=blue,
	filecolor=blue,
	urlcolor=blue,
	citecolor=blue,
}
\usepackage[margin=3cm]{geometry}
\usepackage{mathpazo}
\usepackage{url}
\usepackage{subcaption}
\usepackage{tikz}
\usepackage{pgf}
\usepackage{longtable}
\usepackage{multirow}
\usepackage{graphicx}
\usepackage{cleveref}
\usepackage{bbm}
\usepackage{wrapfig}
\usepackage{mathrsfs}
\usepackage{afterpage}
\usepackage[svgnames]{xcolor}
\usepackage{marvosym}
\usepackage[framemethod=TikZ]{mdframed}
\usepackage{pgfplots}
\usetikzlibrary{3d}

\numberwithin{equation}{section}
\numberwithin{figure}{section}

\newtheorem{theorem}{Theorem}[section]
\newtheorem{thm}{Theorem}[section]
\newtheorem*{thm*}{Theorem}
\newtheorem*{con*}{Conjecture}
\newtheorem{lem}[thm]{Lemma}
\newtheorem{prop}[thm]{Proposition}
\newtheorem{cor}[thm]{Corollary}
\newtheorem{lemma}[thm]{Lemma}
\newtheorem{conj}[thm]{Conjecture}

\theoremstyle{definition}
\newtheorem{defn}[thm]{Definition}
\newtheorem{remark}[thm]{Remark}
\newtheorem{ex}[thm]{Example}
\newtheorem{quest}[thm]{Question}
\newtheorem{obs}[thm]{Observation}
\newtheorem{notation}[thm]{Notation}
\newtheorem{exercise}{Exercise}

\newenvironment{mybox}[1][]{%
\ifstrempty{#1}%
% if condition (without title)
{\mdfsetup{%
    frametitle={%
        %\tikz[baseline=(current bounding box.east),outer sep=0pt]
        %\node[anchor=east,rectangle,fill=RoyalBlue!80] {};
		}
    }%
% else condition (with title)
}{\mdfsetup{%
    frametitle={%
        \tikz[baseline=(current bounding box.east),outer sep=0pt]
        \node[anchor=east,rectangle,fill=RoyalBlue!80,text=white]
        {\strut #1};}%
    }%
}%
% Both conditions
\mdfsetup{%
    innertopmargin=10pt,linecolor=RoyalBlue!80,%
    linewidth=2pt,topline=true,%
    frametitleaboveskip=\dimexpr-\ht\strutbox\relax%
}
\begin{mdframed}[]\relax}{%
\end{mdframed}}

\newenvironment{myboxgreen}[1][]{%
\ifstrempty{#1}%
% if condition (without title)
{\mdfsetup{%
    frametitle={%
        %\tikz[baseline=(current bounding box.east),outer sep=0pt]
        %\node[anchor=east,rectangle,fill=RoyalBlue!80] {};
		}
    }%
% else condition (with title)
}{\mdfsetup{%
    frametitle={%
        \tikz[baseline=(current bounding box.east),outer sep=0pt]
        \node[anchor=east,rectangle,fill=ForestGreen!80,text=white]
        {\strut #1};}%
    }%
}%
% Both conditions
\mdfsetup{%
    innertopmargin=10pt,linecolor=ForestGreen!80,%
    linewidth=2pt,topline=true,%
    frametitleaboveskip=\dimexpr-\ht\strutbox\relax%
}
\begin{mdframed}[]\relax}{%
\end{mdframed}}

\newenvironment{myboxex}[1][1]{%
\vspace{0.25em}
\noindent \begin{minipage}{0.99\textwidth}
	\centering
	\begin{minipage}{#1\textwidth}
		\begin{myboxgreen}[]{}
}{%
		\end{myboxgreen}
	\end{minipage}
\end{minipage}
\vspace{0.25em}
}

\renewcommand{\leq}{\leqslant}
\renewcommand{\geq}{\geqslant}
\newcommand{\N}{\mathbb{N}}
\newcommand{\Z}{\mathbb{Z}}
\newcommand{\Q}{\mathbb{Q}}
\newcommand{\R}{\mathbb{R}}
\newcommand{\C}{\mathbb{C}}
\newcommand{\define}[1]{\textbf{\textit{#1}}}
\newcommand{\WEEK}[1]{%
\hfill Week #1

\vspace{-1em}

\begin{center}
	\rule{\textwidth}{2pt}
\end{center}
\vspace{0.5em}%
}

\setcounter{tocdepth}{2}

\allowdisplaybreaks

\title{Additional Problems}
\author{Joshua Maglione}
\date{\today}

\begin{document}

\maketitle

\section*{Solution 1}

\begin{enumerate}[label=$(\roman* )$]
    \item Let $a\in \mathbb{R}^n$ and $b\in \R$, so that 
    \[ 
        H = \{ v\in \mathbb{R}^n \mid a^{\top}v = b\} 
    \]
    define a hyperplane in $\mathbb{R}^n$. Let $u,v\in H$. Then for all $t\in [0,1]$, 
    \begin{align*}
        a^{\top}(t v + (1 - t) u) &= t a^{\top}v + (1 - t) a^{\top} u \\
        &= tb + (1-t)b = b.
    \end{align*}
    Hence, $tv+(1-t)u\in H$, and therefore the whole line segment between $u$
    and $v$ is in $H$. \qed 
    \item Apply the same argument as $(i)$ except some equalities will be weak
    inequalities. \qed
    \item We'll prove it for infinitely many. Let $I$ be an indexing set and
    suppose $S_i\subseteq \mathbb{R}^n$ is convex for all $i\in I$. Set $S =
    \bigcap_{i\in I}S_i \subseteq \mathbb{R}^n$. Let $u,v\in S$. Then $u,v\in
    S_i$ for all $i\in I$. Since $S_i$ is convex, the line segment connecting
    $u$ and $v$ is in $S_i$. Since this holds for all $i\in I$, the line segment
    connecting $u$ and $v$ is in $S$. Hence, $S$ is convex. \qed
    \item Write 
    \[
        f(S) = \left\{ f(v) ~\middle|~ v\in S \right\} \subseteq \mathbb{R}^b.
    \]
    Let $u,v\in f(S)$. Then there exist $x,y\in S$ such that $u=f(x)$ and
    $v=f(y)$. Since $S$ is convex, we have that $ty + (1-t)x\in S$ for all $t\in
    [0,1]$. Thus, $f(ty + (1-t)x)\in f(S)$. Since $f$ is a linear
    transformation, 
    \[ 
        f(ty + (1-t)x) = tf(y) + (1-t)f(x) =tv + (1-t)u
    \]
    is contained in $f(S)$. Hence, $f(S)$ is convex.\qed
\end{enumerate}

\section*{Solution 2}

\begin{enumerate}[label=$(\roman* )$]
    \item The LP in CF is
    
    \begin{myboxex}[0.75]
        Maximize 
        \begin{align*}
            z &= 3x + 2y 
        \end{align*}
        subject to $x,y,s_1,s_2\geq 0$ and 
        \begin{align*}
            2x - y + s_1 &= 6, \\
            2x + y + s_2 &= 10.
        \end{align*}
    \end{myboxex}

    \item From a theorem from class, the extreme points are the basic feasible
    solutions. From another theorem, the basic solutions are obtained by taking
    all pairs of columns of 
    \[
        A = \begin{bmatrix}
            2 & -1 & 1 & 0 \\ 2 & 1 & 0 & 1
        \end{bmatrix}
    \]
    that form a basis. Every $2$-element subset yields a basis, so we have:
    \begin{align*}
        & \begin{bmatrix}
            4 \\ 2 \\ 0 \\ 0
        \end{bmatrix}, & & \begin{bmatrix}
            5 \\ 0 \\ -4 \\ 0
        \end{bmatrix}, & & \begin{bmatrix}
            3 \\ 0 \\ 0 \\ 4
        \end{bmatrix}, & & \begin{bmatrix}
            0 \\ 10 \\ 16 \\ 0
        \end{bmatrix}, & & \begin{bmatrix}
            0 \\ -6 \\ 0 \\ 16
        \end{bmatrix}, & & \begin{bmatrix}
            0 \\ 0 \\ 6 \\ 10
        \end{bmatrix}.
    \end{align*}
    We have exactly four of these points that are feasible:
    \begin{align*}
        & \begin{bmatrix}
            4 \\ 2 \\ 0 \\ 0
        \end{bmatrix}, & & \begin{bmatrix}
            3 \\ 0 \\ 0 \\ 4
        \end{bmatrix}, & & \begin{bmatrix}
            0 \\ 10 \\ 16 \\ 0
        \end{bmatrix}, & & \begin{bmatrix}
            0 \\ 0 \\ 6 \\ 10
        \end{bmatrix}.
    \end{align*}
    \item The optimal point occurs at $(0,10)^{\top}$.
\end{enumerate}

\section*{Solution 3}

\begin{center}
    \begin{tabular}{|c|ccccccc|c|}
        \hline
        & $x_1$ & $x_2$ & $x_3$ & $x_4$ & $x_5$ & $x_6$ & $x_7$ & \\ \hline
        $x_4$ & $0$ & $0$ & $2$ & $1$ & $5/2$ & $0$ & $0$ & $6/7$ \\ 
        $x_1$ & $1$ & $0$ & $5$ & $0$ & $-3$ & $0$ & $-2$ & $2/7$ \\ 
        $x_6$ & $0$ & $0$ & $3$ & $0$ & $4$ & $1$ & $-4$ & $5/7$ \\ 
        $x_2$ & $0$ & $1$ & $0$ & $0$ & $3/2$ & $0$ & $0$ & $1/7$ \\ \hline
    \end{tabular}
\end{center}

\begin{enumerate}[label=$(\roman*)$]
    \item If $x_5$ is the entering variable, then $\theta$-ratios are $12/35$,
    $-2/21$, $5/28$, and $2/21$ for $x_4$, $x_1$, $x_6$, and $x_2$ respectively.
    Therefore, $x_6$ is the departing variable.
    \item If $x_3$ is the entering variable, then $\theta$-ratios are $3/7$,
    $2/35$, $5/21$, and DNE for $x_4$, $x_1$, $x_6$, and $x_2$ respectively.
    Therefore, $x_1$ is the departing variable.
    \item If $x_7$ is the entering variable, then there is no departing variable
    since the pivotal column is non-positive.
\end{enumerate}

\section*{Solution 4}

Suppose $A\in \mathrm{Mat}_{m\times n}(\mathbb{R})$, $b\in \mathbb{R}^m$, and
$c\in \mathbb{R}^n$. Consider the following LP.

\begin{myboxex}[0.75]
    Maximize 
    \begin{align*}
        z &= c^{\top}x
    \end{align*}
    subject to $x\geq 0$ and 
    \begin{align*}
        Ax &\leq b. 
    \end{align*}
\end{myboxex}

The point 
\[ 
    w = \dfrac{1}{3} u + \dfrac{2}{3} v
\]
is on the line segment between $u$ and $v$, namely, $w = L_{u,v}(2/3)$. We know
that a finite intersection of closed half-spaces is convex, and we know that the
set of feasible solutions is a finite intersection of close half-spaces.
Therefore, $w$ is a feasible solution. \qed 

\section*{Solution 5}

Original:
\begin{center}
    \begin{tabular}{|c|ccccc|c|}
        \hline 
        & $x_1$ & $x_2$ & $x_3$ & $x_4$ & $x_5$ & \\ \hline
        $x_3$ & $2/3$ & $0$ & $1$ & $3/5$ & $0$ & $3/2$ \\ 
        $x_2$ & $3/2$ & $1$ & $0$ & $1$ & $0$ & $5/2$ \\ 
        $x_5$ & $5$ & $0$ & $0$ & $2/9$ & $1$ & $2/3$ \\ \hline
        & $4$ & $0$ & $0$ & $-5$ & $0$ & $7/3$ \\ \hline
    \end{tabular}
\end{center}

\noindent Next:
\begin{center}
    \begin{tabular}{|c|ccccc|c|}
        \hline 
        & $x_1$ & $x_2$ & $x_3$ & $x_4$ & $x_5$ & \\ \hline
        $x_3$ & $-7/30$ & $-3/5$ & $1$ & $0$ & $0$ & $0$ \\ 
        $x_4$ & $3/2$ & $1$ & $0$ & $1$ & $0$ & $5/2$ \\ 
        $x_5$ & $14/3$ & $-2/9$ & $0$ & $0$ & $1$ & $1/9$ \\ \hline
        & $23/2$ & $5$ & $0$ & $0$ & $0$ & $89/6$ \\ \hline
    \end{tabular}
\end{center}


\section*{Solution 6}

Let
\begin{align*}
    A &= \begin{bmatrix}
        2 & 3 & 1 & 0 & 0 \\ 
        -1 & 1 & 0 & 2 & 1 \\
        0 & 6 & 1 & 0 & 3
    \end{bmatrix}, & b &= \begin{bmatrix}
        1 \\ 1 \\ 4
    \end{bmatrix}.
\end{align*}
And set 
\begin{align*}
    x &= \begin{bmatrix}
        1 \\ 0 \\ -1 \\ 1 \\ 0 
    \end{bmatrix}, & y &= \begin{bmatrix}
        0 \\ 2 \\ -5 \\ 0 \\ -1 
    \end{bmatrix}, & z &= \begin{bmatrix}
        0 \\ 0 \\ 1 \\ 0 \\ 1
    \end{bmatrix}.
\end{align*}
Since only 2 entries of $z$ are nonzero, $z$ is not basic. Since 
\[ 
    Ax = \begin{bmatrix}
        1 \\ 1 \\ -1
    \end{bmatrix},
\]
the vector $x$ cannot be basic. Since $Ay=b$, it follows that $y$ is a basic
solution. \qed

\section*{Solution 7}

\begin{center}
    \begin{tabular}{|cccccc|c|}
        \hline 
        $x_1$ & $x_2$ & $x_3$ & $s_1$ & $s_2$ & $s_3$ & \\ \hline
        $1$ & $5$ & $2$ & $0$ & $0$ & $3$ & $20$ \\
        $0$ & $2$ & $4$ & $1$ & $0$ & $-4$ & $6$ \\
        $0$ & $2$ & $-1$ & $0$ & $1$ & $3$ & $12$ \\ \hline 
        $0$ & $-5$ & $-3$ & $0$ & $0$ & $3$ & $12$ \\ \hline
    \end{tabular}
\end{center}
\begin{enumerate}[label=$(\roman*)$]
    \item The basic variables are given by where the pivots are located: $x_1$,
    $s_1$, and $s_2$. Therefore, the corresponding basic feasible solution is
    $(20, 0, 0, 6, 12,0)^{\top}$.
    \item The next table is
    \begin{center}
        \begin{tabular}{|c|cccccc|c|}
            \hline 
            & $x_1$ & $x_2$ & $x_3$ & $s_1$ & $s_2$ & $s_3$ & \\ \hline
            $x_1$ & $1$ & $0$ & $-8$ & $-5/2$ & $0$ & $13$ & $5$ \\
            $x_2$ & $0$ & $1$ & $2$ & $1/2$ & $0$ & $-2$ & $3$ \\
            $s_2$ & $0$ & $0$ & $-5$ & $-1$ & $1$ & $7$ & $6$ \\ \hline 
            & $0$ & $0$ & $7$ & $5/2$ & $0$ & $-7$ & $27$ \\ \hline
        \end{tabular}
    \end{center}
    \item The corresponding basic feasible solution is $(5, 3, 0, 0, 6,
    0)^{\top}$ with basic variables $x_1$, $x_2$, and $s_2$.
\end{enumerate}

\section*{Solution 8}

\begin{myboxex}[0.75]
    Maximize 
    \begin{align*}
        z &= x_1 + 2x_2 + x_3
    \end{align*}
    subject to $x\geq 0$ and 
    \begin{align*}
        3x_1 + x_2 - x_3 &= 15, \\
        8x_1 + 4x_2 - x_3 &= 50, \\
        2x_1 + 2x_2 + x_3 &= 20.
    \end{align*}
\end{myboxex}

We need to add in three artificial variables. For the first phase, we need to solve 

\begin{myboxex}[0.75]
    Maximize 
    \begin{align*}
        z &= -85 - 13x_1 - 7x_2 + x_3 
    \end{align*}
    subject to $x\geq 0$, $y\geq 0$, and 
    \begin{align*}
        3x_1 + x_2 - x_3 + y_1 &= 15, \\
        8x_1 + 4x_2 - x_3 + y_2 &= 50, \\
        2x_1 + 2x_2 + x_3 + y_3 &= 20.
    \end{align*}
\end{myboxex}

\noindent The corresponding tableau is 
\begin{center}
    \begin{tabular}{|c|cccccc|c|}
        \hline 
        & $x_1$ & $x_2$ & $x_3$ & $y_1$ & $y_2$ & $y_3$ & \\ \hline
        $y_1$ & $3$ & $1$ & $-1$ & $1$ & $0$ & $0$ & $15$ \\
        $y_2$ & $8$ & $4$ & $-1$ & $0$ & $1$ & $0$ & $50$ \\
        $y_3$ & $2$ & $2$ & $1$ & $0$ & $0$ & $1$ & $20$ \\ \hline 
        & $13$ & $7$ & $-1$ & $0$ & $0$ & $0$ & $-85$ \\ \hline
    \end{tabular}
\end{center}
The next tableau is 
\begin{center}
    \begin{tabular}{|c|cccccc|c|}
        \hline 
        & $x_1$ & $x_2$ & $x_3$ & $y_1$ & $y_2$ & $y_3$ & \\ \hline
        $y_1$ & $5$ & $3$ & $0$ & $1$ & $0$ & $1$ & $35$ \\
        $y_2$ & $10$ & $6$ & $0$ & $0$ & $1$ & $1$ & $70$ \\
        $x_3$ & $2$ & $2$ & $1$ & $0$ & $0$ & $1$ & $20$ \\ \hline 
        & $15$ & $9$ & $0$ & $0$ & $0$ & $0$ & $-85$ \\ \hline
    \end{tabular}
\end{center}
Therefore, there is no feasible solution.

\end{document}