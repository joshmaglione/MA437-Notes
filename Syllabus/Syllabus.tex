\documentclass[a4paper, 12pt]{article}

\usepackage{enumerate}
\usepackage{hyperref}
\hypersetup{
	colorlinks=true,
	linkcolor=blue,
	filecolor=blue,
	urlcolor=blue,
	citecolor=blue,
}
\usepackage{amsmath}
\usepackage{amsthm}
\usepackage{amssymb}
\usepackage[margin=3cm]{geometry}
\usepackage{mathpazo}
\usepackage{url}
\usepackage[labelformat=simple]{subcaption}
\usepackage{tikz}
\usepackage{pgf}
\usepackage{longtable}
\usepackage{multirow}
\usepackage{graphicx}

\begin{document}
\pagestyle{empty}

\begin{center}
{\Large Mathematics for Decision Making I (MA437)} 

\vspace{0.25cm}

{\large Angela Carnevale \& Joshua Maglione}

\vspace{0.25cm}

Semester 1 (2024)
\end{center}

\vspace{0.5cm}

\begin{description}
    \item[Module information:] \hfill
    \begin{description}
      \item[Coordinates:] \hfill \\ [0.5em]
      % \begin{center}
        \begin{tabular}{rcl}
          Wednesdays & 3:00pm -- 3:50pm & AC201 \\ Thursdays & 1:00pm -- 1:50pm & AMB-G005
        \end{tabular}
      % \end{center}
      \item[Contact:] \hfill \\ \href{mailto:angela.carnevale@universityofgalway.ie}{\texttt{angela.carnevale@universityofgalway.ie}} \\ \href{mailto:joshua.maglione@universityofgalway.ie}{\texttt{joshua.maglione@universityofgalway.ie}}
      \item[Website:] \href{https://universityofgalway.instructure.com/}{\textsf{Canvas}} and \url{https://joshmaglione.com/2024MA437.html} 
    \end{description} 
    \vspace{1cm}
    \item[Topics:] We will cover three fundamental topics of optimization:
    \begin{enumerate} 
      \item the simplex algorithm,
      \item polyhedral geometry,
      \item the notion of duality. 
    \end{enumerate}
    Time permitting we will cover some topics in integer programming.
    \vspace{1cm}
    \item[Assessment:] There is no timed final exam for this module. The overall
    assessment comprises three components:
    \begin{itemize}
      \item an in-class test during either week 6 or 7,
      \item a group project, including a short presentation in week 12,
      \item a ``take-home'' exam, due approximately at the end of the exam period.
    \end{itemize}
    Each of the three components is worth the same, so each part is one third of
    the final mark. The details of these components are not yet set and will be
    announced once they are known.
\end{description}



\end{document}
