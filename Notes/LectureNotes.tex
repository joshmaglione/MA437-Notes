\documentclass[a4paper, 12pt]{article}

\usepackage{amsmath}
\usepackage{amsthm}
\usepackage{amssymb}
\usepackage{enumerate}
\usepackage{hyperref}
\hypersetup{
	colorlinks=true,
	linkcolor=blue,
	filecolor=blue,
	urlcolor=blue,
	citecolor=blue,
}
\usepackage[margin=3cm]{geometry}
\usepackage{mathpazo}
\usepackage{url}
\usepackage{subcaption}
\usepackage{tikz}
\usepackage{pgf}
\usepackage{longtable}
\usepackage{multirow}
\usepackage{graphicx}
\usepackage{cleveref}
\usepackage{bbm}
\usepackage{wrapfig}
\usepackage{mathrsfs}
\usepackage{afterpage}
\usepackage[svgnames]{xcolor}

\numberwithin{equation}{section}
\numberwithin{figure}{section}

\newtheorem{theorem}{Theorem}[section]
\newtheorem{thm}{Theorem}[section]
\newtheorem*{thm*}{Theorem}
\newtheorem*{con*}{Conjecture}
\newtheorem{lem}[thm]{Lemma}
\newtheorem{prop}[thm]{Proposition}
\newtheorem{cor}[thm]{Corollary}
\newtheorem{lemma}[thm]{Lemma}
\newtheorem{conj}[thm]{Conjecture}

\theoremstyle{definition}
\newtheorem{defn}[thm]{Definition}
\newtheorem{remark}[thm]{Remark}
\newtheorem{ex}[thm]{Example}
\newtheorem{quest}[thm]{Question}
\newtheorem{obs}[thm]{Observation}

\renewcommand{\leq}{\leqslant}
\renewcommand{\geq}{\geqslant}
\newcommand{\N}{\mathbb{N}}
\newcommand{\Z}{\mathbb{Z}}
\newcommand{\Q}{\mathbb{Q}}
\newcommand{\R}{\mathbb{R}}
\newcommand{\C}{\mathbb{C}}
\newcommand{\define}[1]{\textit{#1}}

\setcounter{tocdepth}{2}

\allowdisplaybreaks

\title{The Mathematics of Decision Making I}
\author{Joshua Maglione}
\date{\today}

\begin{document}

\maketitle
\tableofcontents

\section{Introduction}

The mathematics of decision making is very closely tied to the field of
mathematical optimization. One of the primary ways mathematics is used to help
guide decisions is by maximizing (or minimizing) specific outcomes subject to a
list of constraints. Mathematical optimization provides the formal tools to
model and solve such problems.

There are many kinds of mathematical optimization. There are two basic types
depending on whether the variables to optimize or discrete or continuous. A few
types of optimization are\footnote{``Program'' is not a computer program but
comes from the United States military's use of the word for training and
logistics schedules.}
\begin{itemize}
	\item Linear Programming,
	\item Integer Programming,
	\item Stochastic programming,
	\item Combinatorial optimization,
	\item Dynamic programming.
\end{itemize}
Unsurprisingly there are many real-world applications; to list a few we have
network optimization, pricing strategy, scheduling, supervised machine learning
training, supply chain optimization, and transportation problems. 

In this module, we will introduce the fundamentals of \textbf{linear
programming}, also called \emph{linear optimization} and \emph{operations
research}, such as the simplex method, polyhedral geometry, and the notion of
duality. Depending on the time, we may also delve into \textbf{integer
programming}.

\subsection{History}

Mathematical optimization has quite an interesting history. In the 17th century,
combinatorial optimization problems were solved using game theory,
combinatorics, and ad hoc methods. In the 19th century, transportation problems
involving post and rail were studied and solved. And in the 20th century with
the two World Wars and rise of the assembly line, operations research took off
developing the mathematics for all kinds of optimization problems. 

One of the most influential figures in mathematical optimization, and linear
programming in particular, is George Dantzig. He was the recipient of the
President's National Medal of Science in 1975 \cite{NSF} and was credited for 
\begin{quote}
	\emph{inventing linear programming and discovering methods that led to wide-scale scientific and technical applications to important problems in logistics, scheduling, and network optimization, and to the use of computers in making efficient use of the mathematical theory.}
\end{quote}
The proof of the simplex method was classified and proved in 1960 by the
Mathematics Division of the RAND Corporation~\cite{SimplexMethod}.

\subsection{Five examples}



\newpage

\bibliography{bibliography} 
\bibliographystyle{abbrv}

\end{document}